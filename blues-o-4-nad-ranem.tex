%r Czwarta nad ranem
%w Czemu cię nie ma na odległość ręki
\documentclass[a5paper]{article}
 \usepackage[english,bulgarian,russian,ukrainian,polish]{babel}
 \usepackage[utf8]{inputenc}
 %\usepackage{polski}
 \usepackage[T1]{fontenc}
 \usepackage[margin=1.5cm]{geometry}
 \usepackage{multicol}
 \setlength\columnsep{10pt}
 \begin{document}
 %\pagenumbering{gobble}


\noindent
\fontsize{12pt}{15pt}\selectfont
\textbf{Blues o czwartej nad ranem} \\
\fontsize{8pt}{10pt}\selectfont
Stare Dobre Małżeństwo \\ \\
\fontsize{10pt}{12pt}\selectfont

\leftskip0.5cm
\begin{tabular}{@{}p{7cm}p{3cm}@{}}
\noindent
Czwarta nad ranem & A (E) \\
Może sen przyjdzie & cis (fis) \\
Może mnie odwiedzisz & D (E) A \\ \\
\end{tabular}

\leftskip0cm
\noindent
\begin{tabular}{@{}p{8cm}p{3cm}@{}}
Czemu cię nie ma na odległość ręki & A E \\
Czemu mówimy do siebie listami & fis cis \\
Gdy ci to śpiewam u mnie pełnia lata & D A \\
Gdy to usłyszysz będzie środek zimy & D E \\
\end{tabular}

\leftskip0cm
\noindent
\begin{tabular}{@{}p{8cm}p{3cm}@{}}
Czemu się budzę o czwartej nad ranem & A E \\
I włosy twoje próbuję ugłaskać & fis cis \\
Lecz nigdzie nie ma twoich włosów & D A \\
Jest tylko blada nocna lampka & D E \\
Łysa śpiewaczka & fis \\ \\
\end{tabular}

\leftskip0cm
\noindent
\begin{tabular}{@{}p{8cm}p{3cm}@{}}
Śpiewamy bluesa bo czwarta nad ranem & A E \\
Tak cicho żeby nie zbudzić sąsiadów & fis cis \\
Czajnik z gwizkiem świruje na gazie & D A \\
Myślał by kto że rodem z Manhatanu & D E \\ \\
\end{tabular}

\leftskip1cm
\noindent
\begin{tabular}{@{}p{7cm}p{3cm}@{}}
Czwarta nad ranem… \\ \\
\end{tabular}

\leftskip0cm
\noindent
\begin{tabular}{@{}p{8cm}p{3cm}@{}}
Herbata czarna myśli rozjaśnia & A E \\
A list twój sam się czyta & fis cis \\
Że można go śpiewać & D \\
Za oknem mruczą bluesa & A \\
Topole z Krupniczej & D E \\ \\
\end{tabular}

\leftskip0cm
\noindent
\begin{tabular}{@{}p{8cm}p{3cm}@{}}
I jeszcze strażak wszedł na solo & A E \\
Ten z mariackiej wieży & fis cis \\
Jego trąbka jak księżyc & D \\
Biegnie nad topolą & A \\
Nigdzie się jej nie spieszy & D E \\ \\
\end{tabular}

\leftskip1cm
\noindent
\begin{tabular}{@{}p{7cm}p{3cm}@{}}
Już piąta & A (E) \\
Może sen przyjdzie & cis (fis) \\
Może mnie odwiedzisz & D A (D E A)
\end{tabular}

\end{document}
