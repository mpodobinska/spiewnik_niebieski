%w Na przełęczy przysiadł wrzesień
%r Do gór, do beskidzkich gór
\documentclass[a5paper]{article}
 \usepackage[english,bulgarian,russian,ukrainian,polish]{babel}
 \usepackage[utf8]{inputenc}
 %\usepackage{polski}
 \usepackage[T1]{fontenc}
 \usepackage[margin=1.5cm]{geometry}
 \usepackage{multicol}
 \setlength\columnsep{10pt}
 \begin{document}
 %\pagenumbering{gobble}


\noindent
\fontsize{12pt}{15pt}\selectfont
\textbf{Tęsknica} \\
\fontsize{8pt}{10pt}\selectfont
A. Wierzbicki \\ \\
\fontsize{10pt}{12pt}\selectfont
\leftskip0cm
\begin{tabular}{@{}p{6.50cm}p{3cm}@{}}
\noindent
Na przełęczy przysiadł wrzesień & e D \\
Śmieje się ukradkiem & CD H7 \\
Skrzydłem kruka włosy czesze & e D \\
Rozczochranym wiatrom & C H7 e \\
Buczynie jej wargi sine & C G \\
Maluje czerwienią & C H7 \\
I korale jarzębinie & e D \\
W bańki cerkwi leje & C H7 e \\ \\
\end{tabular}

\leftskip1cm
\noindent
\begin{tabular}{@{}p{5.50cm}p{3cm}@{}}
Do gór, do beskidzkich gór & e C \\
Zawracamy kroki & D e \\
Przez równin zielony mur & e C \\
Dolin rzecznych krocie & D e \\
Do gór, do beskidzkich gór & e C \\
Zawracamy oczy & D H7 \\
By dojrzeć w buczyny pniach & e C \\
Madonn twarze złote & D e \\ \\
\end{tabular}

\leftskip0cm
\noindent
\begin{tabular}{@{}p{5.50cm}p{3cm}@{}}
Mgły strącając po dolinach \\
Jesień wozem jedzie \\
Znarowione konie spina \\
Worek chleba wiezie \\
I naszym wołaniem \\ 
Zmęczona odchodzi\\
Tylko echo wyprowadza\\
Na rozstajne drogi\\ \\
\end{tabular}

\leftskip1cm
\noindent
\begin{tabular}{@{}p{5.50cm}p{3cm}@{}}
Do gór…
\end{tabular}

\end{document}
