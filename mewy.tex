%w Mewy, białe mewy
%r Żeglarzom wracającym z morza
\documentclass[a5paper]{article}
 \usepackage[english,bulgarian,russian,ukrainian,polish]{babel}
 \usepackage[utf8]{inputenc}
 %\usepackage{polski}
 \usepackage[T1]{fontenc}
 \usepackage[margin=1.5cm]{geometry}
 \usepackage{multicol}
 \setlength\columnsep{10pt}
 \begin{document}
 %\pagenumbering{gobble}


\noindent
\fontsize{12pt}{15pt}\selectfont
\textbf{Mewy} \\
\fontsize{8pt}{10pt}\selectfont
Tomasz Opoka \\ \\
\fontsize{10pt}{12pt}\selectfont
\leftskip0cm
\begin{tabular}{@{}p{8.50cm}p{3cm}@{}}
\noindent
Mewy, białe mewy wiatrem rzeźbione z pian. & e C D e \\
Skrzydlate białe muzy okrętów odchodzących w dal. \\
Kto wam szybować każe przez horyzontu kres, \\
W bezmierne oceany przez sztormu święty gniew? \\ \\
\end{tabular}

\leftskip1cm
\noindent
\begin{tabular}{@{}p{7.50cm}p{3cm}@{}}
Żeglarzom wracającym z morza & C D e \\
Na pamięć przywodzicie dom. & C H7 e \\
Rozbitkom wasze skrzydła niosą & C D e \\
Nadzieję na zbawienny ląd. & C H7 e \\
& e D C H7 \\ \\
\end{tabular}

\leftskip0cm
\noindent
\begin{tabular}{@{}p{9.50cm}p{3cm}@{}}
Ptaki zapamiętane jeszcze z dziecięcych lat, \\
Drapieżnie spadające ze skał na stary Skagerrak. \\
Wiatr czesał grzywy morza, po falach skacząc lekko biegł.\\
Pamiętam tamte mewy, przestworzy słony zew.\\ \\
\end{tabular}

\leftskip1cm
\noindent
\begin{tabular}{@{}p{7.50cm}p{3cm}@{}}
Żeglarzom wracającym z morza…
\end{tabular}

\end{document}
