%w Zawitał czar poezji
\documentclass[a5paper]{article}
 \usepackage[english,bulgarian,russian,ukrainian,polish]{babel}
 \usepackage[utf8]{inputenc}
 %\usepackage{polski}
 \usepackage[T1]{fontenc}
 \usepackage[margin=1.5cm]{geometry}
 \usepackage{multicol}
 \setlength\columnsep{10pt}
 \begin{document}
 %\pagenumbering{gobble}


\noindent
\fontsize{12pt}{15pt}\selectfont
\textbf{A ty tańcz} \\
\fontsize{8pt}{10pt}\selectfont
sł. i muz. Roman Reszter-Fons \\ \\
\fontsize{10pt}{12pt}\selectfont
\leftskip0cm
\begin{tabular}{@{}p{8.5cm}p{3cm}@{}}
\noindent
Zawitał czar poezji & D C \\
Z właścicielem słów i wierszy & G D \\
Rozlano pierwsze butelki & \\
Z wina źródła pociechy & \\
Marek cichutko zaśpiewał & \\
O wiośnie w oknach swej strzechy & \\
Rozpoczął się bal poetów & \\
W mym pokoju gości wielu & \\ \\
\end{tabular}

\leftskip1cm
\noindent
\begin{tabular}{@{}p{8.5cm}@{}}
A ty tańcz moja zasłono \\
I ty papierowy lampionie \\
I tańcz moja koszulo \\
Wietrze owiewaj me dłonie \\ \\
\end{tabular}

\leftskip0cm
\noindent
\begin{tabular}{@{}p{9.5cm}@{}}
Jak gołębie lecące \\
Jak łabędzie po jeziorze płynące \\
Falą łodzie bujane \\
Mały człowiek w kołysce się śmieje \\
Tabun kolorowych wozów \\
Na łące się zatrzymał \\
Kare konie skubią trawę \\
Motyle ganiają się \\ \\
\end{tabular}

\leftskip1cm
\noindent
\begin{tabular}{@{}p{8.5cm}p{3cm}@{}}
A ty…
\end{tabular}

\end{document}
