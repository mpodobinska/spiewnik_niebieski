%w Z halickich połonin dmie złowrogi wiatr
%r Jeszcze trzaska drzewo i ogień śpiewa
\documentclass[a5paper]{article}
 \usepackage[english,bulgarian,russian,ukrainian,polish]{babel}
 \usepackage[utf8]{inputenc}
 %\usepackage{polski}
 \usepackage[T1]{fontenc}
 \usepackage[margin=1.5cm]{geometry}
 \usepackage{multicol}
 \setlength\columnsep{10pt}
 \begin{document}
 %\pagenumbering{gobble}


\noindent
\fontsize{12pt}{15pt}\selectfont
\textbf{Sarajewo} \\
\fontsize{8pt}{10pt}\selectfont
sł. Jaromir Nohavica, tłum. Antoni Muracki \\ \\
\fontsize{10pt}{12pt}\selectfont
\leftskip0cm
\begin{tabular}{@{}p{8.50cm}p{3cm}@{}}
\noindent
Z halickich połonin dmie złowrogi wiatr, & e a/Fis \\
nasz dobytek skromny bystry potok porwał w świat.& H7 e \\
Żeglujemy niebem, jak żurawi sznur, & a / Fis \\
dwa wędrowne ptaki, dwa listy pośród chmur. & H7 e \\ \\
\end{tabular}

\leftskip1cm
\noindent
\begin{tabular}{@{}p{7.50cm}p{3cm}@{}}
Jeszcze trzaska drewno i ogień śpiewa, & e a \\
a już wzrok zachodzi mgłą. & D7/Fis H7 \\
Tam za wzgórze, tam do Sarajewa & H7 e \\ 
idźmy, Bogu miłość wyznać swą. \\ \\
\end{tabular}

\leftskip0cm
\noindent
\begin{tabular}{@{}p{7.50cm}p{3cm}@{}}
Na twej dłoni ksiądz położy moją dłoń,\\
wianek z tamaryszku w spienioną rzuci toń.\\
Spłynie z gór do morza, jak po rzece kry.\\
Chmury na błękicie, a na ziemi my.\\ \\
\end{tabular}

\leftskip1cm
\noindent
\begin{tabular}{@{}p{7.50cm}p{3cm}@{}}
Jeszcze trzaska drewno i ogień śpiewa… \\ \\
\end{tabular}

\leftskip0cm
\noindent
\begin{tabular}{@{}p{7.50cm}p{3cm}@{}}
Z polnego kamienia nasz zbuduję dom,\\
w ociosanych belkach będzie szumiał dąb.\\
Niechaj wszyscy wiedzą, żem ci słowo dał.\\
Dom nasz będzie pewny - na wieki będzie stał.\\\\
\end{tabular}

\leftskip1cm
\noindent
\begin{tabular}{@{}p{7.50cm}p{3cm}@{}}
Jeszcze trzaska drewno i ogień śpiewa ...
\end{tabular}

\end{document}
