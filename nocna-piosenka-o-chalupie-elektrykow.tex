%w Gdy już zgaśnie słońca blask
\documentclass[a5paper]{article}
 \usepackage[english,bulgarian,russian,ukrainian,polish]{babel}
 \usepackage[utf8]{inputenc}
 %\usepackage{polski}
 \usepackage[T1]{fontenc}
 \usepackage[margin=1.5cm]{geometry}
 \usepackage{multicol}
 \setlength\columnsep{10pt}
 \begin{document}
 %\pagenumbering{gobble}


\noindent
\fontsize{12pt}{15pt}\selectfont
\textbf{Nocna piosenka o Chałupie Elektryków} \\
\fontsize{8pt}{10pt}\selectfont
sł. muz. Maciej Płoński \\ \\
\fontsize{10pt}{12pt}\selectfont
\leftskip0cm
\begin{tabular}{@{}p{7.50cm}p{3cm}@{}}
\noindent
Gdy już zgaśnie słońca blask, & D \\
a ostatni obóz zrzuci płaszcz, & G \\
to życie tutaj wtedy rozkwita, & D \\
i aż do rana płynie muzyka & A G D \\ \\

Gdy rozbłyśnie świec naszych blask \\
to właśnie wtedy nastanie czas \\
by gitary i gardła puścić w ruch \\
a biada temu kto ma dobry słuch. \\ \\

W nocy jest właśnie ta chwila, \\
że wpadnie wielce zmęczony turysta, \\
a szczęśliwy będzie dzięki herbacie, \\
i kawałku noclegu na karimacie. \\ \\

Lecz czasem nastanie i świec tych kres, \\
muzykę należy skończyć nieść, \\
trzeba iść spać i potem na śniadanie, \\
bo jutro wieczorem kolejne granie. \\ \\

Tak właśnie żyje nocą nasz dom, \\
śpiewu i gitar mamy że ło, \\
a przede wszystkim to jest takie miejsce, \\
gdzie każdy turysta od razu jest swiętem.
\end{tabular}

\end{document}
