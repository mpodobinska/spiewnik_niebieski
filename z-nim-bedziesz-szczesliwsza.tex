%w Zrozum to co powiem
\documentclass[a5paper]{article}
 \usepackage[english,bulgarian,russian,ukrainian,polish]{babel}
 \usepackage[utf8]{inputenc}
 %\usepackage{polski}
 \usepackage[T1]{fontenc}
 \usepackage[margin=1.5cm]{geometry}
 \usepackage{multicol}
 \setlength\columnsep{10pt}
 \begin{document}
 %\pagenumbering{gobble}


\noindent
\fontsize{12pt}{15pt}\selectfont
\textbf{Z nim będziesz szczęśliwsza} \\
\fontsize{8pt}{10pt}\selectfont
Stare Dobre Małżeństwo \\ \\
\fontsize{10pt}{12pt}\selectfont
\leftskip0cm
\begin{tabular}{@{}p{8.50cm}p{3cm}@{}}
\noindent
Zrozum to co powiem & a e \\
Spróbuj to zrozumieć dobrze & F G \\
Jak życzenia najlepsze, te urodzinowe & F G \\
Albo noworoczne - jeszcze lepiej może & d E \\
O północy gdy składane & F C \\
Drżącym głosem, niekłamane & E \\ \\
\end{tabular}

\leftskip1cm
\noindent
\begin{tabular}{@{}p{7.50cm}p{3cm}@{}}
Z nim będziesz szczęśliwsza, & F C \\
Dużo szczęśliwsza będziesz z nim & d E \\
Ja cóż- włóczęga, niespokojny duch & F C \\
Ze mną można tylko pójść na wrzosowisko & d a \\
I zapomnieć wszystko & a \\
Jaka epoka, jaki wiek, jaki rok & F C G a \\
Jaki miesiąc, jaki dzień & G C \\
I jaka godzina & a \\
Kończy się a jaka zaczyna & F a \\ \\
\end{tabular}

\leftskip0cm
\noindent
\begin{tabular}{@{}p{7.50cm}p{3cm}@{}}
Nie myśl że nie kocham \\
Lub że tylko troche \\
Ja Cie kocham nie powiem no bo nie wypowiem \\
Tak ogromnie bardzo, jeszcze więcej może \\
I dlatego właśnie żegnaj \\
Zrozum dobrze - żegnaj \\ \\
\end{tabular}

\leftskip1cm
\noindent
\begin{tabular}{@{}p{7.50cm}p{3cm}@{}}
Z nim będziesz szczęśliwsza… \\ \\
\end{tabular}

\leftskip0cm
\noindent
\begin{tabular}{@{}p{7.50cm}p{3cm}@{}}
Ze mną można tylko w dali znikać cicho.
\end{tabular}

\end{document}
