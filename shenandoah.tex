%w O Missouri, ty wielka rzeko
\documentclass[a5paper]{article}
 \usepackage[english,bulgarian,russian,ukrainian,polish]{babel}
 \usepackage[utf8]{inputenc}
 %\usepackage{polski}
 \usepackage[T1]{fontenc}
 \usepackage[margin=1.5cm]{geometry}
 \usepackage{multicol}
 \setlength\columnsep{10pt}
 \begin{document}
 %\pagenumbering{gobble}


\noindent
\fontsize{12pt}{15pt}\selectfont
\textbf{Shenandoah} \\
\fontsize{8pt}{10pt}\selectfont
słowa polskie: Andrzej Mendygrał, R. Soliński, muz. oryginalna \\ \\
\fontsize{10pt}{12pt}\selectfont
\leftskip0cm
\begin{tabular}{@{}p{7.50cm}p{3cm}@{}}
\noindent
O Missouri, ty wielka rzeko! & C F C \\
- Ojcze rzek, kto wiek twój zmierzy? & F C \\
Namioty Indian na jej brzegach, & a C \\
- Away, gdy czółno mknie, & C e F \\
\hspace{1cm} poprzez nurt Missouri. & F C G C \\ \\
 
O, Shenandoah jej imię było, \\
I nie wiedziała co to miłość. \\\\
 
Aż przybył kupiec i w rozterce\\
Jej własne ofiarował serce.\\\\
 
Wziął czółno swe i z biegiem rzeki\\
Dziewczynę uwiózł w świat daleki.\\\\
 
A stary wódz rzekł, że nie może\\
Białemu córka wodza ścielić łoże.\\\\
 
Lecz wódka białych wzrok mu mami,\\
Już woojownicy śpią z duchami.\\\\
 
O, Shenandoah, czewony ptaku,\\
Wraz ze mną płyń po życia szlaku.
\end{tabular}

\end{document}
