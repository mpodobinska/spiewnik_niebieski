%w Kiedy noc się w powietrzu zaczyna
%r Ja jestem noc czerwcowa
\documentclass[a5paper]{article}
 \usepackage[english,bulgarian,russian,ukrainian,polish]{babel}
 \usepackage[utf8]{inputenc}
 %\usepackage{polski}
 \usepackage[T1]{fontenc}
 \usepackage[margin=1.5cm]{geometry}
 \usepackage{multicol}
 \setlength\columnsep{10pt}
 \begin{document}
 %\pagenumbering{gobble}


\noindent
\fontsize{12pt}{15pt}\selectfont
\textbf{Noc czerwcowa} \\
\fontsize{8pt}{10pt}\selectfont
sł. Konstanty Ildefons Gałczyński, muz. M. Ochimowska \\ \\
\fontsize{10pt}{12pt}\selectfont
\leftskip0cm
\begin{tabular}{@{}p{7.50cm}p{3cm}@{}}
\noindent
Kiedy noc się w powietrzu zaczyna & D e \\
Wtedy noc jest jak młoda dziewczyna & G D \\
Wszystko cieszy ją i wszystko śmieszy & e \\
Wszystko chciałaby w ręce brać & G D \\ \\
\end{tabular}

\leftskip0cm
\noindent
\begin{tabular}{@{}p{7.50cm}@{}}
Diabeł dużo jej daje w podarku \\
Gwiazd fałszywych z gwiezdnego jarmarku \\
Noc te gwiazdy do uszu przymierza \\
I z gwizdami chciałaby spać \\ \\
\end{tabular}

\leftskip1cm
\noindent
\begin{tabular}{@{}p{6.50cm}p{3cm}@{}}
Ja jestem noc czerwcowa & D e \\
Królowa jaśminowa & G D \\
Zapatrzcie się w moje ręce & e \\
Wsłuchajcie się w śpiewny chód & G D \\ \\
\end{tabular}

\leftskip0cm
\noindent
\begin{tabular}{@{}p{7.50cm}@{}}
Oczy Wam snami dotknę \\
Napoje Wam dam zawrotne \\
I niebo nad Wami rozwinę \\
Jak rulon srebrnych nut \\ \\
\end{tabular}

\leftskip0cm
\noindent
\begin{tabular}{@{}p{7.50cm}@{}}
Ale zanim mur gwizdny ją oplótł \\
Idzie krokiem tanecznym przez ogród \\
Do ogrodu przez senną ulicę \\
Dzwonią nocy ciężkie zausznice \\ \\
\end{tabular}

\leftskip0cm
\noindent
\begin{tabular}{@{}p{7.50cm}@{}}
I przy każdym tanecznym obrocie \\
Szmaragdami błyszczą kołki w płocie \\
Wreszcie do nas pod same okna \\
I tak tańczy, i śpiewa nam \\ \\
\end{tabular}

\end{document}
