\documentclass[a5paper]{article}
 \usepackage[english,bulgarian,russian,ukrainian,polish]{babel}
 \usepackage[utf8]{inputenc}
 %\usepackage{polski}
 \usepackage[T1]{fontenc}
 \usepackage[margin=1.5cm]{geometry}
 \usepackage{multicol}
 \setlength\columnsep{10pt}
 \begin{document}
 %\pagenumbering{gobble}


\noindent
\fontsize{12pt}{15pt}\selectfont
\textbf{Elegia do sztambucha na ścięcie warkocza} \\
\fontsize{8pt}{10pt}\selectfont
sł. Tadeusz Kijonka, muz. Dariusz Cieślak \\ \\
\fontsize{10pt}{12pt}\selectfont
\leftskip0cm
\begin{tabular}{@{}p{9.00cm}p{3cm}@{}}
\noindent
Ścięłaś warkocz - las upadł, już źródło w popiołach & a \\
Przez grzebień opuszczony wieje wiatr elegii \\
Wysechł w pierścieniu kamień… & G \\
Jak się z Tobą zwołam? \\
Jak dojdą przez jezioro znów do siebie brzegi. & a G C G \\\\
\end{tabular}

\leftskip1cm
\noindent
\begin{tabular}{@{}p{8.00cm}p{3cm}@{}} 
Czy po legendzie lata, tamtą poznam teraz? & a G C G \\
Od ogniska - pogańską, z włosami nad wodą. \\
Ścieżki niepokalane, wszystko wiatr zabiera! \\
Ścięłaś warkocz - Las upadł, Mija pierwsza młodość. \\ \\
\end{tabular}

\leftskip0cm
\noindent
\begin{tabular}{@{}p{8.00cm}p{3cm}@{}}
Zejdę z gór na równiny. Chmurom ciąży żałość. \\
Już blisko żniwa deszczu, \\
Już namiot zwinięty. \\
Tyle po nas - co w ściętym warkoczu zostało. \\
Niknący oddech dymu zaprzeszłej legendy. \\\\
 
Zasypiam u ogniska, Zbudzi mnie popiół i rosa. /x3 & a a a G C G 
\end{tabular}

\end{document}
