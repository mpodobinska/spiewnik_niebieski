%w Ty przychodzisz jak noc majowa
\documentclass[a5paper]{article}
 \usepackage[english,bulgarian,russian,ukrainian,polish]{babel}
 \usepackage[utf8]{inputenc}
 %\usepackage{polski}
 \usepackage[T1]{fontenc}
 \usepackage[margin=1.5cm]{geometry}
 \usepackage{multicol}
 \setlength\columnsep{10pt}
 \begin{document}
 %\pagenumbering{gobble}


\noindent
% \setlength{\columnseprule}{0.1pt}
\noindent
\fontsize{12pt}{15pt}\selectfont
\textbf{Poezja} \\
\fontsize{8pt}{10pt}\selectfont
Na Bani, słowa: Władysław Broniewski \\ \\
\fontsize{10pt}{12pt}\selectfont
\leftskip0cm
\begin{tabular}{@{}p{7.50cm}p{3cm}@{}}
\noindent
Ty przychodzisz jak noc majowa, & cis gis \\
biała noc, uśpiona w jaśminie, & A H \\
i jaśminem pachną twoje słowa, & cis gis \\
i księżycem sen srebrny płynie, & A H \\ \\
\end{tabular}

\leftskip0cm
\noindent
\begin{tabular}{@{}p{7.50cm}@{}}
płyniesz cicha przez noce bezsenne \\
- cichą nocą tak liście szeleszczą- \\
szepcesz sny, szepcesz słowa tajemne, \\
w słowach cichych skąpana jak w deszczu \\ \\
\end{tabular}

\leftskip0cm
\noindent
\begin{tabular}{@{}p{7.50cm}@{}}
To za mało! Za mało! Za mało! \\ 
Twoje słowa tumanią i kłamią! \\
Piersiom żywych daj oddech zapału, \\
wiew szeroki i skrzydła do ramion! \\ \\
\end{tabular}

\leftskip0cm
\noindent
\begin{tabular}{@{}p{7.50cm}@{}}
Nam te słowa ciche nie starczą. \\ 
Marne słowa. I błahe. I zimne. \\
Ty masz werbel nam zagrać do marszu! \\
Smagać słowem! Bić pieśnią! Wznieść hymnem! \\ \\
\end{tabular}

\leftskip0cm
\noindent
\begin{tabular}{@{}p{7.50cm}@{}}
Jest gdzieś radość ludzka, zwyczajna, \\
jest gdzieś jasne i piękne życie. \\
Powszedniego chleba słów daj nam \\
i stań przy nas, i rozkaz-bić się! \\ \\
\end{tabular}

\leftskip0cm
\noindent
\begin{tabular}{@{}p{7.50cm}@{}}
Niepotrzebne nam białe westalki, \\
noc nie zdławi świętego ognia \\
bądź jak sztandar rozwiany wśród walki, \\
bądź jak w wichrze wzniesiona pochodnią! \\ \\
\end{tabular}

\leftskip0cm
\noindent
\begin{tabular}{@{}p{7.50cm}@{}}
Odmień, odmień nam słowa na wargach, \\
naucz śpiewać płomienniej i prościej, \\
niech nas miłość ogromna potarga. \\
Więcej bólu i więcej radości! \\ \\ \\ \\ \\
\end{tabular}

\leftskip0cm
\noindent
\begin{tabular}{@{}p{7.50cm}@{}}
Jeśli w pięści potrzebna ci harfa, \\
jeśli harfa ma zakląć pioruny, \\
rozkaż żyły na struny wyszarpać \\ 
i naciągać, i trącać jak struny. \\ \\
\end{tabular}

\leftskip0cm
\noindent
\begin{tabular}{@{}p{7.50cm}@{}}
Trzeba pieśnią bić aż do śmierci, \\
trzeba głuszyć w ciemnościach syk węży. \\
Jest gdzieś życie piękniejsze od nędzy. \\
I jest miłość. I ona zwycięży. \\ \\
\end{tabular}

\leftskip0cm
\noindent
\begin{tabular}{@{}p{7.50cm}@{}}
Wtenczas daj nam, poezjo, najprostsze \\
ze słów prostych i z cichych- najcichsze, \\
a umarłych w wieczności rozpostrzyj \\
jak chorągwie podarte na wichrze. \\
\end{tabular}
\end{document}
