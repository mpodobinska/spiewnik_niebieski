%w Anioły są takie ciche
\documentclass[a5paper]{article}
 \usepackage[english,bulgarian,russian,ukrainian,polish]{babel}
 \usepackage[utf8]{inputenc}
 %\usepackage{polski}
 \usepackage[T1]{fontenc}
 \usepackage[margin=1.5cm]{geometry}
 \usepackage{multicol}
 \setlength\columnsep{10pt}
 \begin{document}
 %\pagenumbering{gobble}


\noindent
\fontsize{12pt}{15pt}\selectfont
\textbf{Bieszczadzkie anioły} \\
\fontsize{8pt}{10pt}\selectfont
Stare Dobre Małżeństwo / sł. Adam Ziemianin, muz. Krzysztof Myszkowski \\ \\
\fontsize{10pt}{12pt}\selectfont
\leftskip0cm
\begin{tabular}{@{}p{9cm}p{3cm}@{}}
\noindent
a \\ \\

Anioły są takie ciche & a \\
Zwłaszcza te w Bieszczadach & G \\                       
Gdy spotkasz takiego w górach & a \\
Wiele z nim nie pogadasz & e \\ \\
  
Najwyżej na ucho ci powie & C G \\
Gdy będzie w dobrym humorze & C F \\
Że skrzydła nosi w plecaku & C G \\
Nawet przy dobrej pogodzie & a e a \\ \\
 
Anioły są całe zielone \\
Zwłaszcza te w Bieszczadach \\
Łatwo w trawie się kryją \\
I w opuszczonych sadach \\ \\
 
W zielone grają ukradkiem \\
Nawet karty mają zielone \\
Zielone mają pojęcie \\
A nawet zielony kielonek \\ \\
\end{tabular}

\leftskip1cm
\noindent
\begin{tabular}{@{}p{8cm}p{3cm}@{}} 
Anioły bieszczadzkie, bieszczadzkie anioły & C G a \\
Dużo w was radości i dobrej pogody & C G a \\
Bieszczadzkie anioły, anioły bieszczadzkie & C G a \\
Gdy skrzydłem cię trącą już jesteś ich bratem & C G a \\ \\
\end{tabular}

\leftskip0cm
\noindent
\begin{tabular}{@{}p{9cm}p{3cm}@{}}
Anioły są całkiem samotne \\
Zwłaszcza te w Bieszczadach \\      
W kapliczkach zimą drzemią \\        
Choć może im nie wypada \\ \\              
 
Czasem taki anioł samotny \\                        
Zapomni dokąd ma lecieć \\           
I wtedy całe Bieszczady \\                 
Mają szaloną uciechę \\ \\   
\end{tabular}

\leftskip1cm
\noindent
\begin{tabular}{@{}p{8.5cm}p{3cm}@{}}
Anioły bieszczadzkie, bieszczadzkie anioły… \\ \\
\end{tabular}

\leftskip0cm
\noindent
\begin{tabular}{@{}p{9.5cm}p{3cm}@{}} 
Anioły są wiecznie ulotne \\
Zwłaszcza te w Bieszczadach \\
Nas też czasami nosi \\
Po ich anielskich śladach \\ \\
 
One nam przyzwalają \\
I skrzydłem wskazują drogę \\
I wtedy w nas się zapala \\
Wieczny bieszczadzki ogień \\ \\
\end{tabular}

\leftskip1cm
\noindent
\begin{tabular}{@{}p{8.5cm}p{3cm}@{}} 
Anioły bieszczadzkie, bieszczadzkie anioły… \\
…Gdy skrzydłem cię musną już jesteś ich bratem \\ \\
 
Anioły bieszczadzkie, bieszczadzkie anioły… \\
…Gdy skrzydłem cię musną już jesteś ich bratem  \\ \\
 
Anioły bieszczadzkie, bieszczadzkie anioły… \\
…Gdy skrzydłem cię musną już jesteś ich bratem
\end{tabular}

\end{document}
