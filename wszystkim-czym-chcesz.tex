%w Ona żarem jest i lodem
\documentclass[a5paper]{article}
 \usepackage[english,bulgarian,russian,ukrainian,polish]{babel}
 \usepackage[utf8]{inputenc}
 %\usepackage{polski}
 \usepackage[T1]{fontenc}
 \usepackage[margin=1.5cm]{geometry}
 \usepackage{multicol}
 \setlength\columnsep{10pt}
 \begin{document}
 %\pagenumbering{gobble}


\noindent
\fontsize{12pt}{15pt}\selectfont
\textbf{Wszystkim czym chcesz} \\
\fontsize{8pt}{10pt}\selectfont
Cezary Makiewicz \\ \\
\fontsize{10pt}{12pt}\selectfont
\leftskip0cm
\begin{tabular}{@{}p{8.00cm}p{3cm}@{}}
\noindent
G C6 D6 C6 G x2 \\ \\

Ona żarem jest i lodem & G C6 \\
słońcem, przejmującym chłodem & G C6 \\
kiedy gniewa się lepiej z drogi zejść & G e D \\
Bo jej gniew to rwąca rzeka & G C6 \\
Trzeba burzę tę przeczekać & G C6 \\
przyjdzie cicho spytać, czy wybaczyć chcesz & G D C \\ \\

Jest Warszawą i Mrągowem \\
każdym miejscem gdzieś po drodze.\\
Kiedy wracam - otwiera zawsze drzwi\\
Jest kroplą deszczu w upalny dzień \\
nocą gdy nadchodzi sen,\\
cichą nutą co długo brzmi\\\\

Nie musisz mi tłumaczyć, & a C \\
że to ma jakiś sens. & G D \\
Spójrz w jej oczy, a zobaczysz & C G \\
ona wszystkim jest czym chcesz. & C e \\
wszystkim czym chcesz & D G \\ \\

Jest niezwykła choć zwyczajna \\
całkiem nieprzewidywalna\\
jej największe wady nie są takie złe.\\
Czasem niespokojną wodą \\
czasem prostą, wiejską drogą \\
jest jedyną, którą chciałbym mieć.
\end{tabular}

\end{document}
