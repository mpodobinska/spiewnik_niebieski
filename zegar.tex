%w Oddam zegar na zawsze w dobre ręce
%r Już mnie tutaj nic nie trzyma, w każdej chwili mogę iść
\documentclass[a5paper]{article}
 \usepackage[english,bulgarian,russian,ukrainian,polish]{babel}
 \usepackage[utf8]{inputenc}
 %\usepackage{polski}
 \usepackage[T1]{fontenc}
 \usepackage[margin=1.5cm]{geometry}
 \usepackage{multicol}
 \setlength\columnsep{10pt}
 \begin{document}
 %\pagenumbering{gobble}


\noindent
\fontsize{12pt}{15pt}\selectfont
\textbf{Zegar} \\
\fontsize{8pt}{10pt}\selectfont
J. Błyszczak \\ \\
\fontsize{10pt}{12pt}\selectfont
\leftskip0cm
\begin{tabular}{@{}p{7.00cm}p{3cm}@{}}
\noindent
Oddam zegar na zawsze w dobre ręce & a D a /G \\
Stary zegar, który po ojcu mam & a D a /G \\
Zegar co bije w moim sercu & d G e7 a7 \\
Zegar co zęby przy mnie zjadł & d G  C7+ \\ \\

Potraciłem, oddałem prawie wszystko \\
Szafę i dwa krzesła wziął nocny stróż \\
Szczęście, że grabarz wziął łopatę \\
Na pewno jej nie odda już \\ \\
\end{tabular}

\leftskip1cm
\noindent
\begin{tabular}{@{}p{6.00cm}p{3cm}@{}} 
Już mnie tutaj nic nie trzyma & F G  C7+ a7 \\
W każdej chwili mogę iść & F G  C7+ \\
Jeszcze tylko zegar oddam ten & F G6 e a \\
Bo za ciężki by go nieść & F G6 d9 \\ \\
\end{tabular}

\leftskip0cm
\noindent
\begin{tabular}{@{}p{7.00cm}p{3cm}@{}}
Oddam zegar w naprawdę dobre ręce \\
Dobry zegar, co czasem rządzi sam \\
Nie trzeba wcale go nakręcać \\
Kto zechce całkiem darmo dam \\ \\
 
Podpaliłem rupiecie na poddaszu\\
Poszedł banknot na ten cel\\
Nic nie mam, co by mnie trzymało\\
Nic, tylko zegar oddać chcę\\ \\
\end{tabular}

\leftskip1cm
\noindent
\begin{tabular}{@{}p{7.00cm}p{3cm}@{}}
Już mnie tutaj...
\end{tabular}

\end{document}
