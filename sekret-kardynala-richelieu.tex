%w Niegdyś przyjaciel dobry mój
\documentclass[a5paper]{article}
 \usepackage[english,bulgarian,russian,ukrainian,polish]{babel}
 \usepackage[utf8]{inputenc}
 %\usepackage{polski}
 \usepackage[T1]{fontenc}
 \usepackage[margin=1.5cm]{geometry}
 \usepackage{multicol}
 \setlength\columnsep{10pt}
 \begin{document}
 %\pagenumbering{gobble}


\noindent
\fontsize{12pt}{15pt}\selectfont
\textbf{Sekret kardynała Richelieu} \\
\fontsize{8pt}{10pt}\selectfont
Czerwony Tulipan \\ \\
\fontsize{10pt}{12pt}\selectfont
\leftskip0cm
\begin{tabular}{@{}p{6.50cm}p{3cm}@{}}
\noindent
Niegdyś przyjaciel dobry mój & e \\
Dziś długów nie oddaje & a \\
Nic złego nie zrobiłem mu & e h \\
A on mnie nie poznaje & e h e \\ \\

Innego znów za uszy & \\
Ciągnąłem do matury & \\
On w dyrektory ruszył & \\
I patrzy na mnie z góry & \\ \\
\end{tabular}

\leftskip1cm
\noindent
\begin{tabular}{@{}p{8.50cm}@{}}
Kardynała Richelieu sekret wam dziś zdradzę \\
Od przyjaciół Boże strzeż z wrogami sobie poradzę x2 \\ \\
\end{tabular}

\leftskip0cm
\noindent
\begin{tabular}{@{}p{9.50cm}@{}}
Raz kumpel brał na raty \\
Adapter radio daczę \\
A ja będąc żyrantem \\
Do dziś te raty płacę \\ \\

Przyjaciel miał trudności \\
Postawiłem go na nogi \\
On teraz mi z wdzięczności \\
Przyprawić pragnie rogi \\ \\
\end{tabular}

\leftskip1cm
\noindent
\begin{tabular}{@{}p{8.50cm}@{}}
Kardynała Richelieu sekret wam dziś zdradzę \\
Od przyjaciół Boże strzeż z wrogami sobie poradzę x2 \\ \\
\end{tabular}

\leftskip0cm
\noindent
\begin{tabular}{@{}p{9.50cm}@{}}
Bo z przyjaciółmi często \\
Podobne są układy \\
Przyjaciel cię roluje \\
Nic na to nie poradzisz \\ \\

Zjawisko dziś powszechne \\
Nie warto się tym smucić \\
Najlepiej się uśmiechnąć \\
I tak sobie zanucić \\ \\
\end{tabular}

\leftskip1cm
\noindent
\begin{tabular}{@{}p{8.50cm}@{}}
Kardynała Richelieu sekret wam dziś zdradzę \\
Od przyjaciół Boże strzeż z wrogami sobie poradzę x2
\end{tabular}

\end{document}
