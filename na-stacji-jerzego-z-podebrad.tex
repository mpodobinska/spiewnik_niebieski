%w Widzimy się co dzień na schodach w metrze
%r A schody jadą, choć mogłyby stać
\documentclass[a5paper]{article}
 \usepackage[english,bulgarian,russian,ukrainian,polish]{babel}
 \usepackage[utf8]{inputenc}
 %\usepackage{polski}
 \usepackage[T1]{fontenc}
 \usepackage[margin=1.5cm]{geometry}
 \usepackage{multicol}
 \setlength\columnsep{10pt}
 \begin{document}
 %\pagenumbering{gobble}


\noindent
\fontsize{12pt}{15pt}\selectfont
\textbf{Na stacji Jerzego z Poděbrad} \\
\fontsize{8pt}{10pt}\selectfont
sł. i muz. Jaromir Nohavica, tłum. Antoni Muracki  \\ \\
\fontsize{10pt}{12pt}\selectfont
\leftskip0cm
\noindent
\begin{tabular}{@{}p{8.5cm}p{3cm}@{}}
kapodaster II próg & C F C  a G C \\ \\
Widzimy się co dzień na schodach w metrze, & C F G \\
gdy ona jedzie na dół, a ja na powierzchnię & C F G \\
Ja wracam z nocnej zmiany, & a \\
ty pracujesz rano & e \\
Ja jestem niewyspany, & F G \\
ty z twarzą zatroskaną & a C \\ \\
\end{tabular}
 
\leftskip1cm
\noindent
\begin{tabular}{@{}p{7.5cm}p{3cm}@{}}
A schody jadą, choć mogłyby stać & F G C \\
na stacji Jerzego z Podebrad & F G C  \\ \\
\end{tabular}
 
\leftskip0cm
\noindent
\begin{tabular}{@{}p{8.5cm}p{3cm}@{}}
Praga o szóstej jeszcze sennie ziewa\\
i tylko my naiwni – robimy co trzeba\\
Ja spieszę się z kliniki,\\
gna do kiosku ona\\
Zmęczone dwa trybiki,\\
dwie wyspy wśród miliona\\ \\
\end{tabular}
 
\leftskip1cm
\noindent
\begin{tabular}{@{}p{8.5cm}p{3cm}@{}}
A schody jadą, choć mogłyby stać… \\ \\
\end{tabular}
 
\leftskip0cm
\noindent
\begin{tabular}{@{}p{8.5cm}p{3cm}@{}}
Choć o tej samej porze - randki są ruchome, \\
bo w tym tandemie każdy jedzie w swoją stronę \\
Ja w lewo, ona w prawo\\
nie ma odwrotu \\
ją czeka Rude pravo\\
a na mnie pusty pokój\\\\
\end{tabular}
 
\leftskip1cm
\noindent
\begin{tabular}{@{}p{8.5cm}p{3cm}@{}}
A schody jadą, choć mogłyby stać… \\ \\
\end{tabular}
 
\leftskip0cm
\noindent
\begin{tabular}{@{}p{8.5cm}p{3cm}@{}}
Na czarodziejskich schodach czuję w sercu drżenie, \\
gdy  kioskareczka Ewa śle mi swe spojrzenie\\
W pospiechu ledwie zdążę\\
szepnąć  - „witam z rana”,\\
bo całowania w biegu\\
surowo się zabrania\\\\
\end{tabular}
 
\leftskip1cm
\noindent
\begin{tabular}{@{}p{8.5cm}p{3cm}@{}}
A schody jadą, choć mogłyby stać…\\\\
\end{tabular}
 
\leftskip0cm
\noindent
\begin{tabular}{@{}p{8.5cm}p{3cm}@{}}
A Praga drzemie i nic jeszcze nie wie\\
o dwojgu zakochanych, zapatrzonych w siebie\\
Już tęsknią nasze włosy\\
w pędzie poplątane\\
do tego, co nas czeka\\
do tego, co nieznane\\\\
\end{tabular}
 
\leftskip1cm
\noindent
\begin{tabular}{@{}p{8.5cm}p{3cm}@{}}
A schody jadą, choć mogłyby stać…
\end{tabular}

\end{document}
