%w Dla zakochanych i bezdomnych
%r Bo to jest zajazd „Pod różą”
\documentclass[a5paper]{article}
 \usepackage[english,bulgarian,russian,ukrainian,polish]{babel}
 \usepackage[utf8]{inputenc}
 %\usepackage{polski}
 \usepackage[T1]{fontenc}
 \usepackage[margin=1.5cm]{geometry}
 \usepackage{multicol}
 \setlength\columnsep{10pt}
 \begin{document}
 %\pagenumbering{gobble}


\noindent
\fontsize{12pt}{15pt}\selectfont
\textbf{Zajazd „Pod różą”} \\
\fontsize{8pt}{10pt}\selectfont
sł. i muz. Zbigniew Kmin, repertuar grupy Nijak \\ \\
\fontsize{10pt}{12pt}\selectfont
\leftskip0cm
\begin{tabular}{@{}p{8.50cm}p{3cm}@{}}
\noindent
Dla zakochanych i bezdomnych & G C G \\
Wędrowców, których tropi pech & G C D \\
Jest jedno miejsce, które szczęście im zapewni & C D G e \\
Da ciepłą strawę, świeży chleb & C D G G7 \\
Jest jedno miejsce, które szczęście im zapewni & C D G e \\
Da ciepłą strawę, świeży chleb & C D G G7 \\ \\
\end{tabular}

\leftskip1cm
\noindent
\begin{tabular}{@{}p{7.50cm}p{3cm}@{}}
Bo to jest zajazd „Pod Różą” & C D \\
Pod różą, czerwoną jak płomień & G e \\
Gdzie wino płynie strumieniami & C D G G7 \\
Przy dźwiękach gitary & C D \\
Czas nam płynie wolno & G e \\
Tu możesz zjeść i wypić z nami & C D G D7 \ \\\
\end{tabular}

\leftskip0cm
\noindent
\begin{tabular}{@{}p{7.50cm}p{3cm}@{}}
Gdy głód cię zdybie gdzieś na szlaku \\
I nie masz sił, by dalej iść \\
Uśmiechnij bracie się i pędź do zajazdu \\
On cię nakarmi, da ci pić \\ \\
 
Choć niewygodne twoje łóżko \\
I na poddaszu pokój ten \\
To czystą pościel masz i w kącie lustro \\
I co noc własny, nowy sen \\ \\
 
Wymarzyć można sobie wszystko \\
To, co do głowy przyjdzie nam \\
I w ciemna noc, kiedy gwiazdy błysną \\
Przyśni się znowu zajazd nam
\end{tabular}

\end{document}
