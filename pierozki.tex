%w Dbając o stan fizyczny i duchowy
%r Nie wiem czy z nimi dojdę pierożków mam mało
\documentclass[a5paper]{article}
 \usepackage[english,bulgarian,russian,ukrainian,polish]{babel}
 \usepackage[utf8]{inputenc}
 %\usepackage{polski}
 \usepackage[T1]{fontenc}
 \usepackage[margin=1.5cm]{geometry}
 \usepackage{multicol}
 \setlength\columnsep{10pt}
 \begin{document}
 %\pagenumbering{gobble}


\noindent
\fontsize{12pt}{15pt}\selectfont
\textbf{Pierożki} \\
\fontsize{8pt}{10pt}\selectfont
Krzysztof Jóźwiak \\ \\
\fontsize{10pt}{12pt}\selectfont
\leftskip0cm
\begin{tabular}{@{}p{8.50cm}p{3cm}@{}}
\noindent
Dbając o stan fizyczny i duchowy & B \\
Z kumplami parę piw wlałem do wnętrza głowy. & B C F7+ B \\
Gdy się już się skończyły świńskie kawały & B \\
I czułem się piękny, mężny i wspaniały. & B C F7+ B \\
Zająłem się potrzebami mego ducha, & B \\
Którego ciągłe nęka burczenie brzucha. & B C F7+ B \\
Zszedłem więc do baru kupić coś ciepłego & B \\
Taniego, dobrego i nieprzystępnego. & B C F7+ B \\
& B F B a F0 \\ \\
\end{tabular}

\leftskip1cm
\noindent
\begin{tabular}{@{}p{7.50cm}p{3cm}@{}}
Nie wiem czy z nimi dojdę pierożko w mam mało. & C a/E/a/d/a \\
Pierożków mam mniej, niż mi się wydawało.& C a/E/a/d/a \\
Pierożków mam mało, znajomych mam wielu. & C a/E/a/d/a \\
Znajomych mam wielu, na drodze do celu. & C a/E/a/d/a \\ \\
\end{tabular}

\leftskip0cm
\noindent
\begin{tabular}{@{}p{9.50cm}p{3cm}@{}}
Pani za barem uśmiecha się ładnie.\\
Pytam ją, co poleci, gdy głód mnie dopadnie.\\
A ona mi na to, że pierożki są smaczne.\\
Że wszyscy je lubią, spróbuję też zacznę.\\
Dużo sobie życzę, trzy razy wspomniałem,\\
a raczej przeciętną porcyję dostałem.\\
Połowa sukcesu, jeszcze wrócic do stołu,\\
Lecz wiele sępów na widzenia polu.\\ \\
\end{tabular}

\leftskip0cm
\noindent
\begin{tabular}{@{}p{9.50cm}p{3cm}@{}}
Już jeden łeb wychyla i zerka ukosem.\\
Plan ma obmyślony, śmieje się pod nosem.\\
I nagle dwa pierożki zabrał mi, łotrzysko.\\
Pewnie gdybym nie zakrył, to zabrałby wszystko.\\
Już mam nadzieję, widzę swoje miejsce,\\
lecz pozycja stracona, obie zajęte ręce.\\
I Cap! dwóch kolegów zabiera po jednym.\\
Nikt się nie zlituje nad losem mym biednym.\\\\
\end{tabular}

\leftskip0cm
\noindent
\begin{tabular}{@{}p{9.50cm}p{3cm}@{}}
Zmierza ku mnie ten, co to ja go nie lubię.\\
Pójdę między stoły, to może go zgubię.\\
Lecz nie minie pięc sekund, gdy się przekonałem,\\
Że z dwojga złego to gorsze wybrałem.\\
Bo dwie głodne panie, mi się ukazały\\
„Czy to aby nie mdłe” bezczelnie zapytały.\\
Nie odmówię, bo czyn to jest godny opryszków\\
Zwłaszcza widząc pary okrągłych, zgrabnych… Oczków.\\ \\
\end{tabular}

\leftskip0cm
\noindent
\begin{tabular}{@{}p{9.50cm}p{3cm}@{}}
Nareszcie dotarłem do mego stolika.\\
Patrzę na mój talerz, Co z z tego wynika?\\
Trzy pierożki zostały, pożarłem je naraz.\\
By mi nie zabrał jakiś paskudne zaraz. (ek)\\
I mimo, że brzuszek mój nienasycony,\\
To mój duch dziwnie czuje się zaspokojony.\\
Trza umieć się cieszyc małymi rzeczami,\\
Wtem się lepiej układa ze swymi duchami.\\
\end{tabular}

\end{document}
