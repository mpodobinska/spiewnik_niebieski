%w W Bukowinie góry w niebie rozstrzępionym
%r I nie mogę znaleźć Bukowiny
\documentclass[a5paper]{article}
 \usepackage[english,bulgarian,russian,ukrainian,polish]{babel}
 \usepackage[utf8]{inputenc}
 %\usepackage{polski}
 \usepackage[T1]{fontenc}
 \usepackage[margin=1.5cm]{geometry}
 \usepackage{multicol}
 \setlength\columnsep{10pt}
 \begin{document}
 %\pagenumbering{gobble}


\noindent
\fontsize{12pt}{15pt}\selectfont
\textbf{Bukowina 1} \\
\fontsize{8pt}{10pt}\selectfont
Wolna Grupa Bukowina / sł. i muz. Wojtek Bellon \\ \\
\fontsize{10pt}{12pt}\selectfont
\leftskip0cm
\begin{tabular}{@{}p{8.3cm}p{3cm}@{}}
\noindent
W Bukowinie góry w niebie postrzępionym & a d7 e7 a7 \\
W Bukowinie rosną skrzydła świętym bukom & a d7 e7 a7 \\
Minął dzień wiatrem z hal rozdzwoniony & C7+ G C7+ a7 \\ \\
\end{tabular}

\leftskip1cm
\noindent
\begin{tabular}{@{}p{7.3cm}p{3cm}@{}}
I nie mogę znaleźć Bukowiny & d7 e7 a7 \\
I nie mogę znaleźć & d7 e7 a7 \\
Chociaż gwiazdy mnie prowadzą ciągle szukam & d7 a7 e7 a7 \\ \\
\end{tabular}

\leftskip0cm
\noindent
\begin{tabular}{@{}p{9.5cm}@{}}
W Bukowinie zarośnięte echem lasy \\
W Bukowinie liść zieleni się i złoci \\
Śpiewa czasem banior ciemnym basem \\ \\
\end{tabular}

\leftskip1cm
\noindent
\begin{tabular}{@{}p{8.5cm}@{}}
I nie mogę znaleźć Bukowiny \\
I nie mogę znaleźć, \\
Choć już szukam godzin krocie i dni krocie. \\ \\
\end{tabular}

\leftskip0cm
\noindent
\begin{tabular}{@{}p{9.5cm}@{}}
W Bukowinie deszczem z chmur opada \\
Okrzyk ptasi zawieszony w niebie \\
Nocka gwiezdną gadkę górom gada \\ \\
\end{tabular}

\leftskip1cm
\noindent
\begin{tabular}{@{}p{8.5cm}@{}}
I nie mogę znaleźć Bukowiny \\
I nie mogę znaleźć \\
Choć mnie woła Bukowina wciąż do siebie. \\ \\
\end{tabular}

\end{document}
