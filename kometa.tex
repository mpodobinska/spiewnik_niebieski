%w Ujrzałem Kometę ze złotym warkoczem
%r O lesie, o trawie, o wodzie
\documentclass[a5paper]{article}
 \usepackage[english,bulgarian,russian,ukrainian,polish]{babel}
 \usepackage[utf8]{inputenc}
 %\usepackage{polski}
 \usepackage[T1]{fontenc}
 \usepackage[margin=1.5cm]{geometry}
 \usepackage{multicol}
 \setlength\columnsep{10pt}
 \begin{document}
 %\pagenumbering{gobble}


\noindent
\fontsize{12pt}{15pt}\selectfont
\textbf{Kometa} \\
\fontsize{8pt}{10pt}\selectfont
sł. Jaromir Nohavica, tłum. Stefan Brzozowski \\ \\
\fontsize{10pt}{12pt}\selectfont
\leftskip0cm
\begin{tabular}{@{}p{8.50cm}p{3cm}@{}}
\noindent
Ujrzałem Kometę ze złotym warkoczem & a a2 a \\
Zaśpiewać jej chciałem, zniknęła mi z oczu & a a2 a \\
Przez chwilę świeciła nad lasem, a potem & d G \\
zostały mi w oczach monety dwie złote & C E7 \\ \\
 
Monety ukryłem w szczelinie pod dębem \\
Gdy kiedyś powróci gdzie indziej już będę \\
Gdzie indziej już będę i duszą, i ciałem \\
Ujrzałem kometę, zaśpiewać jej chciałem \\ \\
\end{tabular}

\leftskip0.5cm
\begin{tabular}{@{}p{7.50cm}p{3.5cm}@{}}
\noindent
O lesie, o trawie, o wodzie & a a d d2 d \\
O śmierci, co po każdego przychodzi & G C E7 \\
O miłości, o zdradzie, o świecie & a a d d2 d \\
O wszystkich ludziach, co żyli tu, na tej planecie. & E E7 a E \\ \\
\end{tabular}

\leftskip0cm
\noindent
\begin{tabular}{@{}p{9.50cm}p{3cm}@{}}
Jak nocne pociągi po niebie mkną gwiazdy \\
Pan Kepler ustalił ich rozkład jazdy \\
Odnalazł wpatrzony w gwiaździste przestrzenie \\
blask tej tajemnicy, co nam jest jak brzemię \\ \\
 
Cień wiecznie tajemnej reguły w przyrodzie \\
Że tylko z człowieka człowiek się narodzi \\
Pniem drzewa z gałęzią wciąż łączy się korzeń \\
Krew naszych nadziei wędruje przestworzem \\ \\
\end{tabular}

\leftskip1cm
\noindent
\begin{tabular}{@{}p{9.50cm}p{3cm}@{}}
O lesie, o trawie, o wodzie… \\ \\
\end{tabular}

\leftskip0cm
\noindent
\begin{tabular}{@{}p{9.50cm}p{3cm}@{}}
Ujrzałem kometę na niebie szerokim, \\
jak relief artysty z minionej epoki \\
Sięgnąłem, by chwycić, zatrzymać przy sobie \\
I naraz poczułem, jak mały jest człowiek \\ \\
 
Jak posąg Dawida wykuty w marmurze \\
wciąż stałem, szukając Komety tam w górze \\
Daremnie czekałem gdy ona powróci \\
mnie już tu nie będzie, kto inny zanuci… \\ \\
\end{tabular}

\leftskip1cm
\begin{tabular}{@{}p{9.50cm}p{3cm}@{}}
\noindent
O lesie, o trawie, o wodzie \\
O śmierci, co po każdego przychodzi \\
O miłości, o zdradzie, o świecie \\
Piosenka ta będzie o nas i Komecie.
\end{tabular}

\end{document}
