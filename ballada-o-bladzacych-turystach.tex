%w Być może, że kiedyś ktoś wpadnie
%r O tych, co dawno dojść powinni
\documentclass[a5paper]{article}
 \usepackage[english,bulgarian,russian,ukrainian,polish]{babel}
 \usepackage[utf8]{inputenc}
 %\usepackage{polski}
 \usepackage[T1]{fontenc}
 \usepackage[margin=1.5cm]{geometry}
 \usepackage{multicol}
 \setlength\columnsep{10pt}
 \begin{document}
 %\pagenumbering{gobble}


\noindent
\fontsize{12pt}{15pt}\selectfont
\textbf{Ballada o błądzących turystach} \\
\fontsize{8pt}{10pt}\selectfont
Leonard Luther \\ \\
\fontsize{10pt}{12pt}\selectfont
\leftskip0cm
\begin{tabular}{@{}p{8.5cm}p{3cm}@{}}
\noindent
Być może, że kiedyś ktoś wpadnie & C d \\
O ile się znajdzie ktoś bystry \\
Na pomysł, by w górach postawić & E7 C \\
Pomnik błądzącego turysty. \\ \\

A kiedy napotkasz ten pomnik & C7 F \\
Turysto prawdziwy bez skazy & C \\
To odrzuć swą pychę i pomyśl & G7 \\
Nadawszy powagę swej twarzy: \\ \\
\end{tabular}

\leftskip1cm
\noindent
\begin{tabular}{@{}p{7.5cm}p{3cm}@{}}
O tych, co dawno dojść powinni, & G7 C \\
A jeszcze wciąż idą & G7 \\
O tych, co po długim marszu & d G7 \\
Wyjściowy ujrzą widok, & C \\
O tych, których kompas skierował & C \\
Na składnicę złomu, & F \\
I o tych, których GOPR sprowadził & G7 \\
Z wycieczki do domu. & C \\ \\

Pomyśl o tych, co dobrym poszli szlakiem & C7 F \\
Ale w złym kierunku, & G7 \\
I o tych co po lesie błądzą & d G7 \\
wołając ratunku, & C \\
I pomny, że się mylił ojciec twój i dziadek & C7 F d \\
Sprawdź, czy ty dobrze idziesz & G7 \\
Na wszelki wypadek. & C \\\\
\end{tabular}

\leftskip0cm
\noindent
\begin{tabular}{@{}p{8.5cm}p{3cm}@{}}
Nie wszyscy błądzący turyści\\
Znaleźli ratunek niestety,\\
Z niektórych już tylko zostały\\
Wyblakłe i ciche szkielety,\\\\

Więc kiedy napotkasz ten szkielet – \\
Turysto prawdziwy bez skazy-\\
To odrzuć swą pychę i módl się\\
Nadawszy powagę swej twarzy: \\ \\
\end{tabular}

\leftskip1cm
\noindent
\begin{tabular}{@{}p{8.5cm}p{3cm}@{}}
O tych, co dawno dojść powinni…
\end{tabular}

\end{document}
