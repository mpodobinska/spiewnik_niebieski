%w Przy małej wiejskiej kapliczce
%r Hej Panie Boże, coś wielkim gazdą nad gazdami
\documentclass[a5paper]{article}
 \usepackage[english,bulgarian,russian,ukrainian,polish]{babel}
 \usepackage[utf8]{inputenc}
 %\usepackage{polski}
 \usepackage[T1]{fontenc}
 \usepackage[margin=1.5cm]{geometry}
 \usepackage{multicol}
 \setlength\columnsep{10pt}
 \begin{document}
 %\pagenumbering{gobble}


\noindent
\fontsize{12pt}{15pt}\selectfont
\textbf{Modlitwa wędrownego grajka} \\
\fontsize{8pt}{10pt}\selectfont
sł. Jan Kasprowicz \\ \\
\fontsize{10pt}{12pt}\selectfont
\leftskip0cm
\begin{tabular}{@{}p{7.50cm}p{3cm}@{}}
\noindent
Przy małej wiejskiej kapliczce, & d C d \\
Stojącej wedle drogi, & a C d \\
Ukląkł rzępoląc na skrzypkach & d C d \\
Wędrowny grajek ubogi. & d a C d \\ \\

Od czasu do czasu grający \\
Bezzębne otwierał wargi, \\
To przekomarzał się z Bogiem, \\
To znów się korzył bez skargi. \\ \\
\end{tabular}

\leftskip1cm
\noindent
\begin{tabular}{@{}p{6.50cm}p{3cm}@{}}
Hej Panie Boże, coś wielkim & g C6 \\
Gazdą nad gazdami, & d \\
Po coś mi dał taką skrzypkę, & C (A7) \\
Co jeno tumani i mami. & d C d \\ \\
\end{tabular}

\leftskip0cm
\noindent
\begin{tabular}{@{}p{7.50cm}p{3cm}@{}}
Spraw to, ażebym na zawsze \\
Umiał dziękować ci Panie, \\
Że sobie rzępolę jak mogę, \\
Że daję Ci co mnie stanie. \\ \\

A jeszcze bardziej chroń mnie \\
I od najmniejszej zawiści, \\
Że są na świecie grajkowie \\
Pełni szumniejszych liści. \\ \\
\end{tabular}

\leftskip1cm
\noindent
\begin{tabular}{@{}p{7.50cm}p{3cm}@{}}
Hej Panie Boże, coś wielkim… \\ \\
\end{tabular}

\leftskip0cm
\noindent
\begin{tabular}{@{}p{7.50cm}p{3cm}@{}}
I niechaj pomnę w mym życiu \\
Czy bliskim, czy też dalekim, \\
Żem człowiek jest przede wszystkim \\
I niczym innym niż człekiem. \\ \\

Spraw w końcu by przy tej kapliczce, \\
Obok tej wiejskiej drogi \\
Klękał i grywał na skrzypkach \\
Wędrowny grajek ubogi. 
\end{tabular}

\end{document}
