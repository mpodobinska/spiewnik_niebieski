%w Kontury dachów pod gasnącym niebem
%r Wracam do nich z mojego miasta
\documentclass[a5paper]{article}
 \usepackage[english,bulgarian,russian,ukrainian,polish]{babel}
 \usepackage[utf8]{inputenc}
 %\usepackage{polski}
 \usepackage[T1]{fontenc}
 \usepackage[margin=1.5cm]{geometry}
 \usepackage{multicol}
 \setlength\columnsep{10pt}
 \begin{document}
 %\pagenumbering{gobble}


\noindent
\fontsize{12pt}{15pt}\selectfont
\textbf{Miejsca pod niebem} \\
\fontsize{8pt}{10pt}\selectfont
Na Bani / sł.: Tom Borkowski, muz.: Agnieszka Rybakautor \\ \\
\fontsize{10pt}{12pt}\selectfont
\leftskip0cm
\begin{tabular}{@{}p{8.50cm}p{3cm}@{}}
\noindent
E H A a E \\ \\

Kontury dachów pod gasnącym niebem & cis gis \\
Niebem, którego niebawem nie będzie & fis A H \\
Życie tam ledwo w zarysie dostrzegam & cis gis \\
Niebacznie mijając je w pędzie & fis A H \\ \\

Miasteczka, gdzie zaczynam podróże & E A \\
Mieściny, w których kończy się droga & Fis A H \\
Gdzie nigdy nie przystaję na dłużej & cis gis \\
Zapomniane przez ludzi, nie Boga & fis A H \\ \\
\end{tabular}

\leftskip1cm
\noindent
\begin{tabular}{@{}p{7.50cm}p{3cm}@{}}
Wracam do nich z mojego miasta & gis A \\
Które stamtąd po trzykroć dalekie & fis A H \\
Które z innego świata wyrasta & gis fis \\
Chociaż na końcu tej samej rzeki & A a H (E) \\ \\
\end{tabular}

\leftskip0cm
\noindent
\begin{tabular}{@{}p{8.50cm}p{3cm}@{}}
Przy sennym rynku pod samym niebem \\
Przelotnie tylko odkrywam wzrokiem \\
Ulice, o których niczego nie wiem \\
Domy, którym nie zajrzę do okien \\ \\

W chwili pomiędzy mną a górami \\
Minąłem ludzi, których nie poznam \\
Nie zdarzy się pewnie nic między nami \\
Niewidzialni dla siebie przechodnie \\ \\
\end{tabular}

\leftskip1cm
\noindent
\begin{tabular}{@{}p{7.50cm}p{3cm}@{}}
Wracam do nich z mojego miasta \\
Które stamtąd po trzykroć dalekie \\
Które z innego świata wyrasta \\
Chociaż na końcu tej samej rzeki \\ \\
\end{tabular}

\leftskip2cm
\noindent
\begin{tabular}{@{}p{6.50cm}p{3cm}@{}}
A jednak wracam i w górę płynę \\
I nagle góry do morza schodzą \\
Od hal do fal wiatr dmuchać zaczyna \\
A mnie do nieba tędy po drodze
\end{tabular}

\end{document}
