%w Kolejna noc Wielkim Wozem wędruje
%r A góry się piętrzą i rosną do nieba
\documentclass[a5paper]{article}
 \usepackage[english,bulgarian,russian,ukrainian,polish]{babel}
 \usepackage[utf8]{inputenc}
 %\usepackage{polski}
 \usepackage[T1]{fontenc}
 \usepackage[margin=1.5cm]{geometry}
 \usepackage{multicol}
 \setlength\columnsep{10pt}
 \begin{document}
 %\pagenumbering{gobble}


\noindent
\fontsize{12pt}{15pt}\selectfont
\textbf{Zapach chleba} \\
\fontsize{8pt}{10pt}\selectfont
Cisza Jak Ta / sł. i muz. M. Borowiec \\ \\
\fontsize{10pt}{12pt}\selectfont
\leftskip0cm
\begin{tabular}{@{}p{8.00cm}p{3cm}@{}}
\noindent
Kolejna noc Wielkim Wozem wędruje & G C9/5 a7 D \\
Ze szczytu Tarnicy można go złapać za koło & G C9/5 a7 D \\
Księżyc granie gór cienką kreską maluje & C D G e \\
Na złoto, zielono i czerwono & C D G \\ \\
\end{tabular}

\leftskip1cm
\noindent
\begin{tabular}{@{}p{7.00cm}p{3cm}@{}}
A góry się piętrzą i rosną do nieba & C h e \\
Bieszczadzkie baśnie nam szepczą do ucha & C D G \\
Zapachem połonin i pieczonego chleba & C h e \\
Śnić się będą, gdy znowu zapanuje plucha. & C D e/G \\ \\
\end{tabular}

\leftskip0cm
\noindent
\begin{tabular}{@{}p{8.00cm}@{}}
Kolejny bar i stół pełen piwa \\
Na ławach zasiadły upadłe anioły \\
W kuflach skrzydlatych zmęczenie odpływa \\
Splatają się pieśni z dźwiękami gitary \\ \\
\end{tabular}

\leftskip1cm
\noindent
\begin{tabular}{@{}p{7.00cm}@{}}
A góry się piętrzą i rosną do nieba \\
Bieszczadzkie baśnie nam szepczą do ucha \\
Zapachem połonin i pieczonego chleba \\
Śnić się będą, gdy znowu zapanuje plucha. \\ \\

e a C h a h C D \\ \\
\end{tabular}

\leftskip0cm
\noindent
\begin{tabular}{@{}p{8.00cm}@{}}
Kolejny szlak pnie się krętą drogą \\
Warkoczem połonin, panien roześmianych \\
Deptany ciężką, ludzką nogą \\
Prowadzi nas do schronisk – rajów obiecanych. \\ \\
\end{tabular}

\leftskip1cm
\noindent
\begin{tabular}{@{}p{7.00cm}@{}}
A góry się piętrzą i rosną do nieba \\
Bieszczadzkie baśnie nam szepczą do ucha \\
Zapachem połonin i pieczonego chleba \\
Śnić się będą, gdy znowu zapanuje plucha.
\end{tabular}

\end{document}
