%w Jedni wolą brunetki, drudzy wolą blond loki
\documentclass[a5paper]{article}
 \usepackage[english,bulgarian,russian,ukrainian,polish]{babel}
 \usepackage[utf8]{inputenc}
 %\usepackage{polski}
 \usepackage[T1]{fontenc}
 \usepackage[margin=1.5cm]{geometry}
 \usepackage{multicol}
 \setlength\columnsep{10pt}
 \begin{document}
 %\pagenumbering{gobble}


\noindent
\fontsize{12pt}{15pt}\selectfont
\textbf{Ballada na trzy chwyty} \\
\fontsize{8pt}{10pt}\selectfont
Gwiazdy z zagranicy / sł. Jan Szewczak, Karol Łyda, muz. Karol Łyda\\ \\
\fontsize{10pt}{12pt}\selectfont
\leftskip0cm
\begin{tabular}{@{}p{9.5cm}p{3cm}@{}}
\noindent
%\hspace{1cm}La la laj laj… \\ \\

Jedni wolą brunetki, drudzy wolą blond loki, & D G \\
a mi się marzy dziewczę, co stawia duże kroki. & A D \\
Jedni lubią z uśmiechem, inni celują w ponure, \\
a mi marzy dziewczę, co lubi iść pod górę. \\ \\

Nie musi mieć kusych dekoltów, tym bardziej chodzić w mini, \\
byle lubiła podróż szlakami wysokogórskimi. \\
Nie będzie na szlaku stękać, gdy podkład się rozmaże, \\
na nogach jej będą treki, a chodzić będzie w polarze. \\ \\

\hspace{1cm}La la laj laj… \\ \\

Nie wiem czy taka wizja jest dziś prawdziwie możliwa, \\ 
lecz zawsze mieć będzie w plecaku miejsce na dwa piwa. \\
Nieważne jej pochodzenie i nieważne ile ma wzrostu, \\
nie będą jej nigdy przeszkadzać kędziory górskiego zarostu. \\ \\

Nie będzie jej nigdy przeszkadzać, mimo dziewczęcej urody, \\
że musi spać na podłodze i znowu zabrakło wody. \\
Gdy rano na kacu myślę kurczę, już zginę \\
ona wyciąga pół litra schowane na czarną godzinę. \\ \\

\hspace{1cm}La la laj laj… \\ \\

A w kwestii charakteru, to żeby nie narzekała \\
chyba, że na pogodę, albo że góra za mała. \\
I żeby lubiła Stachurę, czasem czytała Leśmiana, \\
znała a-dur na gitarze i biła ze mną do rana.\\ 

Do kuchni nie mam wymagań, podobnie jest u mnie z wiarą. \\
Mogła by wierzyć we mnie i umieć gotować makaron (Lubella!) \\ \\

\hspace{1cm}La la laj laj… \\ \\

Dziewczę wejdzie do schroniska i zobaczy nas we dwóch, \\
w rękach naszych dwie gitary, ponad stół opasły brzuch. \\
Smutny koniec tej ballady, niewesoły i normalny, \\
na realia nie da rady, to ucieczka każdej panny. \\
Pozostanie nam schronisko, dwie gitary, góry, my \\
nasze bębny, kastaniety, plecak, pieśń, a w myślach Ty!
\end{tabular}

\end{document}
