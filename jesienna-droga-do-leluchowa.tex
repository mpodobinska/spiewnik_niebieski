%w Droga przez sam środek jesieni
%r Snuje się na drodze do Leluchowa
\documentclass[a5paper]{article}
 \usepackage[english,bulgarian,russian,ukrainian,polish]{babel}
 \usepackage[utf8]{inputenc}
 %\usepackage{polski}
 \usepackage[T1]{fontenc}
 \usepackage[margin=1.5cm]{geometry}
 \usepackage{multicol}
 \setlength\columnsep{10pt}
 \begin{document}
 %\pagenumbering{gobble}


\noindent
\fontsize{12pt}{15pt}\selectfont
\textbf{Jesienna droga do Leluchowa} \\
\fontsize{8pt}{10pt}\selectfont
sł. Jerzy Harasymowicz, muz. Tomasz Fojgt \\ \\
\fontsize{10pt}{12pt}\selectfont
\leftskip0cm
\begin{tabular}{@{}p{8.00cm}p{3cm}@{}}
\noindent
& d d - C \\ \\
\end{tabular}

\leftskip0cm
\noindent
\begin{tabular}{@{}p{8.00cm}p{3cm}@{}}
Droga przez sam środek jesieni & d g \\
Pod liści złotym tunelem & B C d \\
Dopala się wzgórze nad Popradem & d g \\
Pachnie polskim zasuszonym zielem & B C d \\ \\
\end{tabular}

\leftskip1cm
\noindent
\begin{tabular}{@{}p{7.00cm}p{3cm}@{}}
Snuje się na drodze do Leluchowa & B F \\
Poezji Lelum-Polelum & g A7 \\ 
Słonecznik zza chmur się wychyla & B F \\
Dzień dobry Ci mój przyjacielu & g a B C \\ \\
& d d - C \\ \\
\end{tabular}

\leftskip0cm
\noindent
\begin{tabular}{@{}p{8.00cm}p{3cm}@{}} 
Snuje się tu moja duszyczka sowia & d g \\
Zbiera przyprawy do swojej baśni & B C d \\
Czerwony dereń października & d g \\
Piołuny goryczki i dziurawca & B C d \\ \\
\end{tabular}

\leftskip1cm
\noindent
\begin{tabular}{@{}p{7.00cm}p{3cm}@{}}
Droga przez sam środek jesieni & D A \\
Syczą w Popradzie liście jak iskry & h G \\
Rzucam i rzucam czarne kiście guseł & D A h G \\
Na wiatrów wir płomienisty & D A B C \\ \\
& d d - C
\end{tabular}

\end{document}
