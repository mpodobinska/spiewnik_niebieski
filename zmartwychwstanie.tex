%w Szedł Jezus przez Beskidy
%r I głośne Alleluja górami się niosło
\documentclass[a5paper]{article}
 \usepackage[english,bulgarian,russian,ukrainian,polish]{babel}
 \usepackage[utf8]{inputenc}
 %\usepackage{polski}
 \usepackage[T1]{fontenc}
 \usepackage[margin=1.5cm]{geometry}
 \usepackage{multicol}
 \setlength\columnsep{10pt}
 \begin{document}
 %\pagenumbering{gobble}


\noindent
\fontsize{12pt}{15pt}\selectfont
\textbf{Zmartwychwstanie} \\
\fontsize{8pt}{10pt}\selectfont
Dom o Zielonych Progach / sł. K. J. Węgrzyn \\ \\
\fontsize{10pt}{12pt}\selectfont
\leftskip0cm
\begin{tabular}{@{}p{6.00cm}p{3cm}@{}}
\noindent
Szedł Jezus przez Beskidy & e \\
W wiosenny poranek & h \\
Choć nogi miał i ręce & C \\
Krwawo rozorane & G D \\ \\

Choć bok miał opuchnięty \\
I blady cierpieniem \\
Choć trzy dni leżał ciężkim \\
Przybity kamieniem \\ \\

To w ręce niósł wysoko \\
Chorągiew radości \\
By wszyscy zrozumieli \\ 
Posłanie miłości \\ \\

Szedł Jezus przez Beskidy \\
A góry klękały \\
Aby grzbiety pochylić \\
W imię Bożej chwały \\ \\
\end{tabular}

\leftskip1cm
\noindent
\begin{tabular}{@{}p{5.00cm}@{}}
I głośne Alleluja! \\
Górami się niosło \\
Bo gdzie chwilę przystanął \\
Budził życie wiosną \\ \\
\end{tabular}

\leftskip0cm
\noindent
\begin{tabular}{@{}p{6.00cm}@{}}
A echo jak z trombity \\
Wracało nad gronie \\
Aby króla przywitać \\
W cierniowej koronie \\ \\

Zaś gdy rany zabliźnił \\
Ludziom i przyrodzie \\
Rzucał ziarno nadziei \\
Aby rosło co dzień. \\ \\
\end{tabular}

\leftskip1cm
\noindent
\begin{tabular}{@{}p{5.00cm}@{}}
I głośne Alleluja! \\
Górami się niosło \\
Bo gdzie chwilę przystanął \\
Budził życie wiosną 
\end{tabular}

\end{document}
