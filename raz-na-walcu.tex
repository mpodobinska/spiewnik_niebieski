\documentclass[a5paper]{article}
 \usepackage[english,bulgarian,russian,ukrainian,polish]{babel}
 \usepackage[utf8]{inputenc}
 %\usepackage{polski}
 \usepackage[T1]{fontenc}
 \usepackage[margin=1.5cm]{geometry}
 \usepackage{multicol}
 \setlength\columnsep{10pt}
 \begin{document}
 %\pagenumbering{gobble}


\noindent
\fontsize{12pt}{15pt}\selectfont
\textbf{Raz na walcu} \\
\fontsize{8pt}{10pt}\selectfont
Marek Andrzejewski \\
\fontsize{10pt}{12pt}\selectfont

\leftskip0.5cm
\begin{tabular}{@{}p{6.50cm}p{3cm}@{}}
\noindent
Raz na walcu, raz pod walcem & A \\
Życie polega na walce & E \\
Na walce lub na wyścigu & h A \\
Raz pod dźwigiem, raz na dźwigu & E A \\
-- x2 -- \\ \\
\end{tabular}

\leftskip0cm
\noindent
\begin{tabular}{@{}p{7.50cm}p{3cm}@{}}
Żeby płaszczyć i rozgniatać & E \\
Walec świetnie się nadaje & A E A \\
Kto na walec się nie wdrapał & E \\
Ten już nijak nie odstaje & A E h A \\ \\
\end{tabular}

\leftskip1cm
\noindent
\begin{tabular}{@{}p{6.50cm}p{3cm}@{}}
Raz na walcu, raz pod walcem… \\ \\
\end{tabular}

\leftskip0cm
\noindent
\begin{tabular}{@{}p{6.50cm}p{3cm}@{}}
Kto pod walcem, ten przeminie \\
W mieście, w cieście, w jednym placku \\
Ciasno jest więc przy drabinie\\
I miejsc nie ma już w kabinie \\ \\
\end{tabular}

\leftskip1cm
\noindent
\begin{tabular}{@{}p{6.50cm}p{3cm}@{}}
Raz na walcu, raz pod walcem… \\ \\
\end{tabular}

\leftskip0cm
\noindent
\begin{tabular}{@{}p{6.50cm}p{3cm}@{}}
Raz pod walcem, raz na walcu \\
życie polega na szmalcu \\
Na szmalcu lub na etacie \\
Raz na stracie, raz w senacie, o ja cie! \\ \\
\end{tabular}

\leftskip1cm
\noindent
\begin{tabular}{@{}p{6.50cm}p{3cm}@{}}
Raz na walcu, raz pod walcem… \\ \\
\end{tabular}

\end{document}
