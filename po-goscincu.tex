%w Po gościńcu podobłocznym
\documentclass[a5paper]{article}
 \usepackage[english,bulgarian,russian,ukrainian,polish]{babel}
 \usepackage[utf8]{inputenc}
 %\usepackage{polski}
 \usepackage[T1]{fontenc}
 \usepackage[margin=1.5cm]{geometry}
 \usepackage{multicol}
 \setlength\columnsep{10pt}
 \begin{document}
 %\pagenumbering{gobble}


\noindent
\fontsize{12pt}{15pt}\selectfont
\textbf{Po gościńcu} \\
\fontsize{8pt}{10pt}\selectfont
Bez Jacka / sł. na podstawie „Newa poezją płynąca”, muz. Z. Stefański, J.H. Chrząstek\\ \\
\fontsize{10pt}{12pt}\selectfont
\leftskip0cm
\begin{tabular}{@{}p{7.50cm}p{3cm}@{}}
\noindent
Po gościńcu podobłocznym, & d a \\
przez ostatek, rzek parowy. & d a \\
Nadal niedorzecznie kroczy & F E \\
jakiś człowiek nietypowy. & F E \\ \\
\end{tabular}

\leftskip0cm
\noindent
\begin{tabular}{@{}p{7.50cm}p{3cm}@{}}
Za plecami śmiech ma tylko, & d a \\
twarz niezmiernie uśmiechniętą, & C \\
a przez napar i omyłkę, & d a \\
nie je, nie śpi, któryś dzień tam. & d a \\ \\
\end{tabular}

\leftskip0cm
\noindent
\begin{tabular}{@{}p{7.50cm}@{}}
Uwiązany do postronka \\
przy nim pies ogonem majda, \\
ktoś tam chlapnie za nim z okna: \\
serwus, jakże ci ciamajdo. \\ \\
\end{tabular}

\leftskip0cm
\noindent
\begin{tabular}{@{}p{7.50cm}@{}}
Nie opuszcza go ochota \\
ciągle śmiać się do każdego \\ 
ej idiota, oj idiota, \\
dzieci wrzeszczą w ślad czeredą. \\ \\
\end{tabular}

\leftskip0cm
\noindent
\begin{tabular}{@{}p{7.50cm}@{}}
W chatach go, ciosanych z bali, \\
staruszkowie, od wiek wieka \\
na noc chętnie zapraszali, \\
za świętego mieli człeka. \\ \\
\end{tabular}

\leftskip0cm
\noindent
\begin{tabular}{@{}p{7.50cm}@{}}
Pytać im się nie wypada, \\
czy on stąd, czy przybysz z dali, \\
kto zapalił las za sadem? \\
A nikt, sam się tak wciąż pali.
\end{tabular}

\end{document}
