%w W rozstrzelanej chacie
%r Święty Mikołaju, opowiedz jak tu było
\documentclass[a5paper]{article}
 \usepackage[english,bulgarian,russian,ukrainian,polish]{babel}
 \usepackage[utf8]{inputenc}
 %\usepackage{polski}
 \usepackage[T1]{fontenc}
 \usepackage[margin=1.5cm]{geometry}
 \usepackage{multicol}
 \setlength\columnsep{10pt}
 \begin{document}
 %\pagenumbering{gobble}


\noindent
\fontsize{12pt}{15pt}\selectfont
\textbf{Ballada o świętym Mikołaju} \\
\fontsize{8pt}{10pt}\selectfont
SRTA / sł. i muz. Andrzej Wierzbicki \\ \\
\fontsize{10pt}{12pt}\selectfont
\leftskip0cm
\begin{tabular}{@{}p{8cm}p{3cm}@{}}
\noindent
W rozstrzelanej chacie & a G E E \\
Rozpaliłem ogień & a G a a \\
Z rozwalonych pieców & a G E E \\
Pieśni wyniosłem węgle & F E7 \\
Naciągnąłem na drzazgi gontów & a C \\
Błękitną płachtę nieba & d E \\
Będę malować od nowa & a d C E a \\
Wioskę w dolinie & d E a (G) \\ \\
\end{tabular}

\leftskip1cm
\noindent
\begin{tabular}{@{}p{7cm}p{3cm}@{}}
Święty Mikołaju, & C G \\
Opowiedz jak tu było & C E \\
Jakie pieśni śpiewano & a d C E a \\
Gdzie się pasły konie & d E a (G) \\ \\
\end{tabular}

\leftskip0cm
\noindent
\begin{tabular}{@{}p{8.5cm}p{3cm}@{}}
A on nie chce gadać, \\
Ze mną po polsku \\
Z wypalonych źrenic \\
Tylko deszcze płyną \\
Hej, ślepcze, nauczę swoje \\
Dziecko po łemkowsku \\
Będziecie razem żebrać \\
W malowanych wioskach \\ \\
\end{tabular}

\leftskip1cm
\noindent
\begin{tabular}{@{}p{8.5cm}p{3cm}@{}} 
Święty Mikołaju… \\
\end{tabular}

\end{document}
