%w Ojczyzną moją jest muzyka
\documentclass[a5paper]{article}
 \usepackage[english,bulgarian,russian,ukrainian,polish]{babel}
 \usepackage[utf8]{inputenc}
 %\usepackage{polski}
 \usepackage[T1]{fontenc}
 \usepackage[margin=1.5cm]{geometry}
 \usepackage{multicol}
 \setlength\columnsep{10pt}
 \begin{document}
 %\pagenumbering{gobble}


\noindent
\fontsize{12pt}{15pt}\selectfont
\textbf{Saskia} \\
\fontsize{8pt}{10pt}\selectfont
Bez Jacka / sł. K.I. Gałczyński muz. J.H .Chrząstek, Z .Stefański. \\ \\
\fontsize{10pt}{12pt}\selectfont
\leftskip0cm
\begin{tabular}{@{}p{9.50cm}@{}}
\noindent
e a e a C G D e a e a C G D \\ \\
\end{tabular}

\leftskip0cm
\noindent
\begin{tabular}{@{}p{9.50cm}p{3cm}@{}}
Ojczyzną moją jest muzyka, & e a \\
a Ty jesteś jak nuta rzewna z którą na ustach, & e a C \\
po latach wraca się w muzykę jak do domu & D e \\
i tylko nie trwóż się Saskia, bo możesz mi być & e a e \\
tak bardzo potrzebna, że chyba tylko śmierci, & e a C \\
ale już Cię nie oddam nikomu. & D e \\ \\
\end{tabular}

\leftskip0cm
\noindent
\begin{tabular}{@{}p{9.50cm}p{3cm}@{}}
Jeżeli wszystkie niebiosa i wszystkie w nich Serafiny & e C D e \\
krzykiem tęsknoty wybłagam, by się spełniła twa chwała, & e C D e \\
jeżeli powiem Ci więcej, że jesteś ponad rubiny, & e C D e \\
o jedno proszę cię Saskia, nie bądź zarozumiała. & e C D e \\ \\
\end{tabular}

\leftskip0cm
\noindent
\begin{tabular}{@{}p{9.50cm}@{}}
Saskia cokolwiek się stanie, \\
złotą chmurą zostaniesz w legendzie, \\
bo wypisane jest ogniem na ścianie, \\
że cię nikt tak jak ja kochał nie będzie. \\
Sięgnij po nieśmiertelność Saskia, \\
tak jak się sięga po jabłko rumiane ogromne. \\
Jabłko szmaragd przeminą, jak komar brzęczy gitara, \\
lecz ja to w książkę zamknę. \\
Saskia, uśmiechnij się do mnie.
\end{tabular}

\end{document}
