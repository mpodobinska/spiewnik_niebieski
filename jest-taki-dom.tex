%w Odnajdziesz go gdzie wiatr wplątany
\documentclass[a5paper]{article}
 \usepackage[english,bulgarian,russian,ukrainian,polish]{babel}
 \usepackage[utf8]{inputenc}
 %\usepackage{polski}
 \usepackage[T1]{fontenc}
 \usepackage[margin=1.5cm]{geometry}
 \usepackage{multicol}
 \setlength\columnsep{10pt}
 \begin{document}
 %\pagenumbering{gobble}


\noindent
\fontsize{12pt}{15pt}\selectfont
\textbf{Jest taki dom} \\
\fontsize{8pt}{10pt}\selectfont
Jarek Kochanowski, Szczyt Możliwości \\ \\
\fontsize{10pt}{12pt}\selectfont
\leftskip0cm
\begin{tabular}{@{}p{8.50cm}p{3cm}@{}}
\noindent
Odnajdziesz go gdzie wiatr wplątany & C F \\
W przeniczny zapach chleba & C a \\
Gdzie każde drzewo szumi swoją pieśń & C F G \\
Gdzie rozzłoconą drogę wita wiejska strzecha & F G C a \\
Codziennie rada z deszczem spory wieść & F G C \\ \\
										 & C F C \\ \\
\end{tabular}

\leftskip0cm
\noindent
\begin{tabular}{@{}p{8.50cm}p{3cm}@{}}
Tam w środku dnia zmęczeni słońcem & \\
Zbierzemy się w ogrodzie & \\
Gdzie wśród jabłoni stoi stary stół & \\
Pomilczmy wtedy chwwilę by słuchać w bzu powodzi & \\
Jak lato śpiewa miodnym głosem pszczół & \\ \\
									   & C F C \\ \\
\end{tabular}

\leftskip1cm
\noindent
\begin{tabular}{@{}p{7.50cm}p{3cm}@{}}
Jest taki dom, dom w którym & F G \\
Każdy znajdzie swoje miejsce & C a \\
Tam chleb i miłość ten sam mają smak & F G C7 \\
Tam w sosen pniach wiatr wiersze składa & F G \\
W kominie świerszcz coś gada & F G \\ 
Życzliwym niech nam będzie szlak & F G \\
							   & F G C x2 \\ \\
\end{tabular}

\leftskip0cm
\noindent
\begin{tabular}{@{}p{8.50cm}@{}}
Tam wiatr z komina wróży przyszłość \\
Na niebie kreśląc słowa \\
Przy stole zawsze wolne miejsce jest \\
Hej drogo dobra drogo \\
Do takich drzwi doprowadź \\
Przez które zwykły człowiek mógłby przejść
\end{tabular}

\end{document}
